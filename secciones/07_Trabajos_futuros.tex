\chapter{Trabajos futuros}\label{chap:7trabajosfuturos}

En este capítulo se exploran las posibles vías de expansión y mejoras para el proyecto en el futuro.

\section{Mejoras en la interfaz de usuario}
La línea de comandos como interfaz de IPFShare es adecuada para un prototipo, aunque esto no se traslada a un
producto final que busque ser usado por usuarios no técnicos. Por tanto, una mejora obvia sería la implementación de una interfaz gráfica de usuario.

Esta interfaz gráfica se podría proveer mediante una aplicación web o de escritorio. En ambos casos se podría aprovechar la implementación actual de IPFShare, ya que
esta se ha desarrollado en NodeJS, lo que permite que se pueda ejecutar en el navegador o en el escritorio mediante Electron\cite{BuildCrossplatformDesktop}.

Esta mejora no es trivial, ya que el soporte de OrbitDB en el navegador requiere de servidores WebRTCStar \cite{OrbitDBrtc2023} para el correcto descubrimiento de nodos en la red. Esto es algo que se podría solucionar con la implementación de un servidor propio, esto implicaría añadir un componente centralizado que
haría que el sistema no fuera completamente autocontenido y distribuido.

\section{Uso de streams para el consumo de contenido en memoria}
Dentro de la implementación actual se han integrado dos comandos que exploran este concepto. Estos son \texttt{ipfshare cat} y \texttt{ipfshare ls}.
Estos comandos permiten consumir contenido de IPFS en memoria mediante streams, lo que habilita el uso de estos comandos en scripts y otras herramientas.

El uso de concepto en otros comandos merece la pena ser explorado por las ventajas de consumo de disco y memoria que ofrece.

\section{Mejoras en el sistema de notificaciones}
En la implementación actual las notificaciones son creadas por los propios usuarios para sí mismos al llegar nuevas entradas
del ShareLog dirigidas a estos. En ciertas ocasiones el evento sobre el que se realizan estas acciones no se dispara, por lo que
las notificaciones pueden pasar desapercibidas. La solución a este problema consiste en el uso de bases de datos locales
en las que se añaden las comparticiones que se tienen del ShareLog, y se comprueba si ya se han mostrado previamente.

El algoritmo de comprobación de duplicados que implementa esta funcionalidad:
\begin{minted}{typescript}

async ensureLocalUpToDate() { 
    await this.localSharedWithMeStore.load()
    await this.localSharedWithOthersStore.load()
    await this.store.load()
    const localSharedWithMeEntries = this.localSharedWithMeStore.iterator().collect()
    const localSharedWithOtherEntries = this.localSharedWithOthersStore.iterator().collect()
    this.store.iterator().collect().forEach(
        (entry) => { 
            (async () => {
                if (entry.payload.value.senderId === this._orbitdb.id &&
                    localSharedWithOtherEntries.filter(async (localEntry) => {
                        return localEntry.payload.value.shareCID.toString() === entry.payload.value.shareCID.toString()
                    }).length == 0) {
                    await this.localSharedWithOthersStore.add(entry.payload.value)
                }
                if (entry.payload.value.recipients.includes(this._orbitdb.id)
                    &&
                    entry.payload.value.senderId !== this._orbitdb.id
                    &&
                    localSharedWithMeEntries.filter(
                        (localEntry) => localEntry.payload.value.shareCID.toString() === entry.payload.value.shareCID.toString()
                    ).length === 0
                ) {
                    await this.localSharedWithMeStore.add(entry.payload.value)
                    const value: ShareLogEntry = entry.payload.value
                    logger.info('New share available, ${JSON.stringify(value)}')
                    notify(value)
                }
            })()
        })
}
\end{minted}

Como se puede observar este método tiene una complejidad temporal de $O(n^2)$. Esto se debe a que en el peor de los casos, para cada entrada en \texttt
{this.store} (n entradas), se ejecuta una operación de filtrado tanto en \texttt{localSharedWithOtherEntries} como en
\\\texttt{localSharedWithMeEntries} (en el peor de los casos, cada operación de filtrado puede recorrer todos los elementos, que son de tamaño n). Por
lo tanto, la complejidad total es proporcional al cuadrado del número de elementos, es decir, $O(n^2)$. No es una implementación eficiente, pero es la única que se ha encontrado que funciona correctamente, es por ello que se incluye en esta sección.


\section{Mejoras en el sistema de encriptación}

El algoritmo de encriptación que IPFShare utiliza es AES-256-CBC, que es seguro y comúnmente usado. Sin embargo, no es la mejor opción para su uso específico en IPFShare. AES-256-CBC es un algoritmo de cifrado en bloque que no puede paralelizarse, como resultado, el cifrado de archivos grandes puede resultar lento.


\section{Integración de un sistema de ficheros en tiempo real sobre OrbitDB}
Como ya se comentó en el la sección \ref{sec:sailplane}, Sailplane ha implementado un sistema de ficheros montado sobre OrbitDB. Esto es algo que se podría implementar en IPFShare ampliando enormemente las capacidades y casos de uso que IPFShare cubre actualmente.

Se podría ampliar la implementación del store de Sailplane para que este soporte la encriptación de los archivos, y así poder ofrecer un sistema de ficheros encriptado capaz de sincronizarse en tiempo real y automáticamente entre los usuarios que tengan la instancia del store.

\section{Integración de un sistema de ficheros en tiempo real sobre IPFS}

En la anterior sección se ha discutido la posibilidad de usar OrbitDB como la base sobre la que montar un sistema de ficheros en tiempo real.
Sin embargo, también se podría implementar un sistema de ficheros en tiempo real sobre IPFS. Actualmente existen esta funcionalidad en IPFS
mediante el uso del MFS, pero no incorpora ningún tipo de encriptación ni sistema de control de acceso.

Se podría manipular el MFS manualmente para recrear sus funcionalidades pero añadiendo una capa de encriptación por la que pasarían los archivos antes de ser añadidos al MFS. A la hora de la consumición de estos archivos se deberían desencriptar antes de ser consumidos.
