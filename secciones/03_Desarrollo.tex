\chapter{Desarrollo de IPFShare}\label{chap:3desarrollo}
% Capítulo dedicado a describir el desarrollo del Trabajo realizado. De acuerdo con el tutor, este capítulo puede tener distintas estructuras, e incluso pueden existir varios capítulos.
% Todos los capítulos deben empezar en una página nueva.
% Los apartados dentro de los capítulos se numeran de forma jerárquica, pero siempre deben estar alineados al margen izquierdo. Ejemplo:

\section{Casos de uso}

\begin{minted}{typescript}
import OrbitDB from "orbit-db"
// eslint-disable-next-line @typescript-eslint/ban-ts-comment
// @ts-ignore
import AccessController from "orbit-db-access-controllers/interface"
import DocumentStore from "orbit-db-docstore"
import { IdentityProvider } from "orbit-db-identity-provider"

export interface RegistryEntry {
  peerId: string
  orbitdbIdentity: string // DID
  username: string // alias
}

export abstract class Registry<S, DocType> {
    abstract accessController: AccessController
    abstract store: S
    abstract open(): Promise<void> 
    abstract create(): Promise<void>
    abstract replicate(): Promise<void>
    abstract close(): Promise<void>
    abstract addUser(user: DocType): Promise<void>
    abstract getUser(entryId: string): Promise<DocType | undefined>
    abstract updateUser(entryId: string, updates: Partial<DocType>): Promise<void>
    abstract searchUsers(queryFn: (entry: DocType) => boolean): Promise<DocType[]>
    abstract deleteUser(entryId: string): Promise<void>
}

\end{minted}

\section{Objetivos}


\section{Requisitos}


\section{Tecnologías}
\subsection{Tecnologías propuestas}
\subsection{Tecnologías usadas}
\section{Arquitectura del sistema}

\label{sec:arquitectura_del_sistema}

\section{Implementación}

\subsection{Backend (Electron)}
\subsection{Frontend (React)}
