\chapter{Evaluación de la implementación}\label{chap:5evaluacion}

En este capítulo se evaluará la implementación realizada en base a los requisitos definidos en el capítulo \ref{chap:4desarrollo}: '\nameref{chap:4desarrollo}'.

Para ello se ha creado un escenario de uso real con distintos usuarios en varios lugares del mundo. Este escenario se ha utilizado para realizar una serie de pruebas que permitan evaluar el sistema desarrollado. Para ello se han usado varios servidores proporcionados por Github Codespaces.
Estos servidores corren instancias de IPFShare emulando un caso de uso real con usuarios en distintas ubicaciones geográficas.

Las pruebas realizadas son:
\begin{itemize}[noitemsep,after=\vspace{-0.4\baselineskip}]
    \item \textbf{Registro e interconexión:} se comprueba que los usuarios pueden registrarse en el sistema, a medida que cada usuario se registra
          el Registry debe actualizar su lista de usuarios, esto debe verse reflejado en todos los clientes.
    \item \textbf{Compartición de un archivo con un usuario específico}: un usuario comparte un archivo con un usuario específico, este debe recibir
          una notificación y poder descargar el archivo.
    \item \textbf{Compartición de un archivo con todos los usuarios registrados}: un usuario comparte un archivo con todos los usuarios registrados,
          estos deben recibir una notificación y poder descargar el archivo.
    \item \textbf{Compartición de un archivo por parte de un usuario, intento de descarga por parte de un usuario no autorizado}: debe fallar la descarga al estar este usuario no autorizado.
    \item \textbf{Intento de manipulación del Registry - modificación y eliminación de una entrada}: se intenta modificar y eliminar una entrada del Registry por parte de una entidad no autorizada, debe fallar.
    \item \textbf{Intento de un usuario de tener varias entradas en el Registry:} se intenta registrar un usuario con un DID que ya está registrado, debe
          fallar.
    \item \textbf{Comparticiones realizadas por el usuario}: El usuario debe poder acceder a la lista de archivos que ha compartido.
    \item \textbf{Comparticiones recibidas por el usuario}: El usuario debe poder acceder a la lista de archivos que ha recibido.
\end{itemize}

No se han incluido pruebas de comandos en relación a grupos debido a que no se ha implementado esta funcionalidad. Aunque cabe destacar que bajo la 
infraestructura actual podrían ser fácilmente implementadas en un futuro.

Estas pruebas están documentadas en un vídeo que se puede encontrar en el siguiente enlace: \url{https://youtu.be/2Z3Q4Z3Z8ZM}.