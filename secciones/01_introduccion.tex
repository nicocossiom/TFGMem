\chapter{Introducción}\label{chap:1introduccion}
%%---------------------------------------------------------
El presente Trabajo de Fin de Grado (TFG) se centra en el desarrollo de un sistema de intercambio de ficheros basado en
IPFS (InterPlanetary File System)\cite{IPFSPowersDistributed}.
\\A continuación, se describen las motivaciones y necesidades que han llevado a la realización de este proyecto.
\section{Motivación y necesidad}

El desarrollo de un sistema de intercambio de ficheros basado en IPFS se encuentra en la confluencia de varias tendencias
tecnológicas y sociales que están dando forma al futuro de la web. En particular, este proyecto se relaciona estrechamente
con el avance hacia la \textit{Web3}\cite{Web32023}, una visión de un internet más descentralizado, seguro y resistente a la censura.
En esta sección, exploraremos cómo un sistema de intercambio de archivos encaja en este nuevo panorama y por qué es relevante para el progreso de la Web3.

Los servicios de almacenamiento y compartición de archivos actuales, como Google Drive, Dropbox, Microsoft OneDrive y otros
proveedores de almacenamiento en la nube son servicios centralizados. Pese a ser  populares y ampliamente utilizados debido
a su facilidad de uso, accesibilidad y confiabilidad, presentan ciertos problemas y limitaciones. Los usuarios dependen de una
sola entidad para almacenar y gestionar sus archivos, lo que puede generar problemas si la empresa experimenta fallos técnicos,
cambia sus políticas de uso, o se convierte en el objetivo de un ataque cibernéticos maliciosos.
Además, esto otorga a estas empresas un gran poder sobre los datos de los usuarios, lo que puede conducir a problemas de privacidad y control de la información.

Otras alternativas como FTP (File Transfer Protocol) ofrecen una mayor autonomía y control sobre los archivos, pero también
tienen inconvenientes. FTP es un protocolo que permite la transferencia de archivos
entre un cliente y un servidor a través de una red. FTP carece de robustas medidas de seguridad modernas, puede ser vulnerable
a ataques y requiere un mayor conocimiento técnico y esfuerzo para su configuración y mantenimiento.
\\En resumen, a pesar de la mayor autonomía y control directo que FTP
puede ofrecer, no es comparable con un servicio en la nube en términos de seguridad, facilidad de uso y eficiencia de costos.
Esto es teniendo en cuenta los conocimientos y requisitos del usuario promedio de un servicio de estas características.

La arquitectura detrás de este tipo de servicios se basa en el modelo cliente-servidor. En este modelo,
un servidor central almacena la información sobre la lista de nodos y recursos disponibles en la red y es vital para el
funcionamiento del sistema. Esto facilita encontrar rápidamente los nodos o recursos disponibles, pero el sistema es relativamente
vulnerable en términos de fallos o ataques y la escalabilidad está limitada debido a la presión sobre el elemento central \cite{vybochPeertopeerProtocolsFile2017}.

La alternativa a estos servicios centralizados es el uso de tecnologías \textit{peer-to-peer} (de igual a igual en español).
Una aplicación peer-to-peer (p2p) es un tipo de red donde no existen clientes ni servidores fijos,
sino una serie de nodos que actúan como iguales y pueden funcionar tanto como clientes como servidores entre sí.

Existen varias tecnologías p2p que permiten compartir archivos entre usuarios sin necesidad de un proveedor central, el
más famoso y conocido siendo BitTorrent\cite{BitTorrentProtocol}. Sin embargo, estas tecnologías no son adecuadas para el intercambio de archivos entre
usuarios no conocidos, ya que requieren que los usuarios confíen en que los archivos que se comparten son los que se anuncian.

Esto es algo que resuelve el Inter Planetary File System (IPFS).
El Sistema de Archivos Interplanetario es un sistema de archivos distribuido que busca conectar todos los dispositivos
al mismo sistema de archivos. En cierto modo, IPFS es similar a la Web, aunque podría verse como una sola red
BitTorrent, intercambiando objetos dentro de un repositorio Git.
\\En otras palabras, IPFS permite guardar y acceder a bloques de
datos identificados por su contenido, no por su ubicación, y que se pueden transferir rápidamente entre los nodos. Además, IPFS
usa estos bloques para crear enlaces que también se basan en el contenido, no en una dirección que apunta a una ubicación donde se puede encontrar el contenido.
Esto forma un grafo dirigido acíclico generalizado de Merkle (Merkle DAG), una estructura de datos sobre la que se puede
construir sistemas de archivos versionados, cadenas de bloques e incluso una Web Permanente. IPFS combina una tabla hash
distribuida, intercambio de bloques incentivado y espacio de nombres autocertificante, sin puntos únicos de falla ni necesidad
de confianza entre los nodos que la forman\cite{benetIPFSContentAddressed2014}.


En este proyecto se usará IPFS como bloque central, sobre el que construirá el sistema previamente descrito.


\section{Objetivos y alcance del proyecto}
El objetivo principal de este proyecto es el desarrollo de un sistema de intercambio de ficheros basado en IPFS, mediante una aplicación de escritorio.
Este sistema debe permitir a los usuarios compartir archivos de forma segura y confiable, sin necesidad de ningún proveedor
central de ningún tipo.
\\Debe integrar capacidades  de encriptación y control de acceso para garantizar la seguridad de los
archivos compartidos. La integración de cuentas de usuario, con la posibilidad de hacer grupos, elegir contactos con los que
compartir, se propone como algo imprescindible para lograr un sistema autocontenido y sin necesidad de herramientas externas
para su uso. Por último se debe integrar un sistema de notificaciones para el que los usuarios puedan recibir avisos de nuevos
archivos compartidos, o de cambios en los archivos compartidos.

Para lograr esto se han cumplido los siguientes objetivos:
\begin{itemize}

      \item Investigar sobre IPFS y su funcionamiento para entender cómo funciona el protocolo
            y cómo se puede utilizar para el sistema propuesto.
      \item Investigar sobre el ecosistema en torno a IPFS, con objetivo de comprender
            la madurez y viabilidad de esta tecnología, así como de las herramientas basadas en esta
            que se pueden utilizar para el sistema propuesto.
      \item Diseñar una arquitectura para el sistema de intercambio en torno a las tecnologías y herramientas seleccionadas.
      \item Implementación de un prototipo funcional del sistema propuesto.
      \item Analizar la viabilidad de IPFS en base a la experiencia obtenida en el desarrollo del prototipo.
      \item Analizar posibles mejoras y ampliaciones del sistema propuesto.

\end{itemize}

Por tanto pese a que el objetivo principal es el desarrollo de un sistema de intercambio de ficheros basado en IPFS,
también se realizará una labor de divulgativa sobre  IPFS y su ecosistema, con el objetivo de comprender esta
tecnología y su viabilidad como alternativa a muchas de las tecnologías actuales.


\section{Estructura de la memoria}
En este capítulo se ha introducido el proyecto, explicando las motivaciones y necesidades que han llevado a su realización.

En el capítulo \ref{chap:2contexto}: '\nameref{chap:2contexto}' se pone en situación el estado actual de tecnologías relacionadas con el proyecto, tanto alternativas
como otras implementaciones que usen IPFS u otras tecnologías similares que cumplan parcial o completamente con los objetivos del proyecto.
Al comienzo de este capítulo también se explica brevemente la historia de internet y su evolución hasta el presente.
La razón de ser de esta sección se debe a la necesidad de poner en situación el porqué detrás de la dominancia de ciertos
protocolos que han guiado el modelo de internet actual, y que han llevado a la necesidad de alternativas como IPFS.

Dentro de este capítulo se explica el funcionamiento de IPFS, tratando los siguientes temas: arquitectura interna, funcionamiento,
ecosistema y herramientas relacionadas. Con esta sección se busca dar una visión general de esta tecnología y su ecosistema para
poder entender el sistema propuesto.

En el capítulo \ref{chap:3estadodelarte}: '\nameref{chap:3estadodelarte}' se lleva a cabo un breve análisis
de algunas otras implementaciones que usan IPFS o tecnologías similares, así como de otras alternativas a IPFS que cumplen
parcial o completamente con los objetivos del proyecto.

El capítulo \ref{chap:4desarrollo}: '\nameref{chap:4desarrollo}' se centra en el desarrollo del sistema propuesto. Este se ha estructurado en en:
\begin{itemize}
      \item \textbf{Requisitos del sistema:} se explica el funcionamiento deseado del sistema.
      \item \textbf{Diseño del sistema:} se presenta la arquitectura y diseño propuestos en este proyecto, así como las herramientas utilizadas.
      \item \textbf{Implementación:} se explica la implementación realizada, así como las decisiones tomadas durante el desarrollo. Esta sección
            incluye partes de código relevantes para entender la implementación realizada.
\end{itemize}

En el capítulo \ref{chap:5evaluacion}: '\nameref{chap:5evaluacion}' se realiza una
serie de pruebas del sistema desarrollado. Para ello se ha creado un escenario de uso real con distintos usuarios en varios lugares del mundo.

El capítulo \ref{chap:6resultados}: '\nameref{chap:6resultados}' se analiza el resultado obtenido del desarrollo del proyecto. Se contrastará
el resultado con los objetivos propuestos y con servicios de transferencia de archivos centralizados.

Sobre el alcance del proyecto, en el capitulo \ref{chap:7trabajosfuturos}: '\nameref{chap:7trabajosfuturos}' se explora las posibles vías de expansión y mejoras para el proyecto en el futuro.
También se expresan las esperanzas y expectativas para el crecimiento y posible impacto del mismo.
